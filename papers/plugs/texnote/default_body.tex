%DO_NOT_MODIFY{
%This file is the starting point for all notes
%You can edit the template of this file using: papers texnote edit_template -B
\documentclass{article}
%All texnote much share the same style so that we can compile them all together without problem
%You are free to edit the style file using: papers texnote edit_template -S
\usepackage{INFO} %The style location is automatically filled
\begin{document}
%}
%HEADER{
%This part is the header, you can modify it as you wish. It will be removed when compiling higher level notes 
%TITLE, AUTHOR, YEAR, ABSTRACT will be automatyically replaced with respect to the associated bibfile thanks to the \autofill{*FIELD*}{} marker.
\begin{center}
    \Large{\textbf{\autofill{TITLE}{Title not found}}} \\ [0.2cm]
    \small{\textsc{\autofill{AUTHOR}{Author(s) not found}}} \\ [0.2cm]
    \normalsize{\textsc{\autofill{YEAR}{Year not found}}} \\ [1cm]
\end{center}
\begin{abstract}
    \autofill{ABSTRACT}{Abstract not found}
\end{abstract}
%Write your notes below
%Do not use \section{} or \subsection{} as they may be source of problems when concatenating notes.
%}





%DO_NOT_MODIFY{
%You can only cite papers added in you repo.
%To update the bib file with latest papers info: papers texnote generate_bib
\bibliographystyle{INFO} %The bibstyle is automatically filled (default is ieeetr)
%You can change the bibliography style in the config file : .papersrc
%[texnote]
%bib_style = plain
\bibliography{INFO} %The bibliography location is automatically filled
\end{document}
%}
